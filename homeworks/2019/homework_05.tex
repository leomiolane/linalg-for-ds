\documentclass[11pt,nocut]{article}

\usepackage{../latex_style/packages}
\usepackage{../latex_style/notations}

\title{\vspace{-2.0cm}%
	Optimization and Computational Linear Algebra for Data Science\\
Homework 5: Orthogonal matrices, eigenvalues and eigenvectors}
%\author{Léo \textsc{Miolane} \ $\cdot$ \ \texttt{leo.miolane@gmail.com}}
\date{\vspace{-1cm}Due on October 8, 2019}
\setcounter{section}{5}

\begin{document}
\maketitle
%\noindent\textbf{Rules:}
\centerline{\pgfornament[width=13cm]{89}}
{\small
	\begin{itemize}
		\item Unless otherwise stated, all answers must be mathematically justified.
		\item Partial answers will be graded. 
		\item You can work in groups but each student must write his/her own solution based on his/her own understanding of the problem. Please list on your submission the students you work with for the homework (this will not affect your grade).
		\item Problems with a $(\star)$ are extra credit, they will not (directly) contribute to your score of this homework. However, for every $4$ extra credit questions successfully answered your lowest homework score get replaced by a perfect score.
		\item If you have any questions, feel free to contact me (\texttt{lm4271@nyu.edu}) or to stop at the office hours.
	\end{itemize}
}
\vspace{-0.4cm}
\centerline{\pgfornament[width=13cm]{89}}
\vspace{0.5cm}


\begin{problem}[1 points]
	Is the following matrix diagonalizable?
	$$
	M= 
	\begin{pmatrix}
		1 & \pi^2 \\
		0 & 1
	\end{pmatrix}.
	$$
\end{problem}

\vspace{1mm}

\begin{problem}[3 points]
	Let $S$ be a subspace of $\R^n$ and let $P_S$ be the matrix of the orthogonal projection onto $S$. Let $M = \Id_n - 2 P_S$.
	\begin{enumerate}[label=\normalfont(\textbf{\alph*})]
		\item Show that the matrix $M$ is orthogonal.
		\item Show that if $\lambda \in \R$ is an eigenvalue of $M$, then $\lambda = 1$ or $\lambda=-1$.
		\item Show that $M$ is diagonalizable.
\end{problem}

\vspace{1mm}

\begin{problem}[3 points]
	Let $A \in \R^{n \times n}$ be a symmetric matrix.
	\begin{enumerate}[label=\normalfont(\textbf{\alph*})]
		\item Show that if $v_1, v_2 \in \R^n$ are two eigenvectors of $A$ associated to some eigenvalues $\lambda_1 \neq \lambda_2$ ($Av_1 = \lambda_1 v_1$ and $Av_2 = \lambda_2 v_2$), then $v_1 \perp v_2$.
		\item Show that if $A$ is diagonalizable, then there exists an orthonormal basis $(u_1, \dots, u_n)$ of eigenvectors of $A$.
	\end{enumerate}
\end{problem}

\vspace{1mm}

\begin{problem}[3 points]
	Let $A \in \R^{n \times n}$ be a diagonalizable matrix. Let $(v_1, \dots, v_n)$ be a basis of $\R^n$ consisting of eigenvectors of $A$, and let $(\lambda_1, \dots, \lambda_n)$ be the associated eigenvalues. Assume that
	$$
	\lambda_1 > |\lambda_i| \qquad \text{for all} \ i \in \{2, \dots, n\}.
	$$
	We consider the following algorithm:
	\begin{itemize}
		\item Initialize $x_0 \in \R^n$.
		\item Perform the updates: $x_{t+1} = \frac{A x_t}{\|A x_t\|}$.
	\end{itemize}
	\begin{enumerate}[label=\normalfont(\textbf{\alph*})]
		\item Show that for all $t \geq 1$, 
			$$
			x_t = \frac{A^t x_0}{\|A^t x_0\|}.
			$$
		\item Assume that $x_0$ is a unit vector ($\|x_0\|=1$) whose direction is chosen uniformly at random (this basically means that all the possible directions for $x_0$ are equally likely to be chosen). Let $(\alpha_1, \dots, \alpha_n)$ be the coordinates of $x_0$ in the basis $(v_1, \dots, v_n)$. Explain why we can be sure that $\alpha_1 \neq 0$. You do not have to do a rigorous proof of that, just give an intuitive argument.
		\item Show that
			$$
			x_t \xrightarrow[t \to \infty]{} \frac{\alpha_1 v_1}{\|\alpha_1 v_1\|}
			\qquad \text{and} \qquad
			\|A x_t\| \xrightarrow[t \to \infty]{} \lambda_1.
			$$
	\end{enumerate}
\end{problem}

\vspace{1mm}


\begin{problem}[$\star$]
	Let $A \in \R^{n \times n}$ be a symmetric matrix. Define the function
	$$
	\begin{array}{cccc}
		f:& \R^n \setminus \{0\} & \to & \R \\
				& x & \mapsto & \displaystyle \frac{x^{\sT} A x}{x^{\sT} x}.
	\end{array}
	$$
	Show that $f$ has a maximum at some $x_{\star} \in \R^n \setminus \{0\}$ and that $x_{\star}$ verifies
	$$
	A x_{\star} = \lambda x_{\star}, \qquad \text{where} \quad \lambda = f(x_{\star}).
	$$
\end{problem}
\vspace{1cm}
\centerline{\pgfornament[width=7cm]{87}}

%\bibliographystyle{plain}
%\bibliography{./references.bib}
\end{document}
