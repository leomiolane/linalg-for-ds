\documentclass[11pt,nocut]{article}

\usepackage{../../latex_style/packages}
\usepackage{../../latex_style/notations}

\title{\vspace{-2.0cm}%
	Optimization and Computational Linear Algebra for Data Science\\
Homework 8: SVD, linear algebra and graphs}
%\author{Léo \textsc{Miolane} \ $\cdot$ \ \texttt{leo.miolane@gmail.com}}
\date{\vspace{-1cm}Due on November 8, 2020}
\setcounter{section}{8}

\begin{document}
\maketitle
%\noindent\textbf{Rules:}
\centerline{\pgfornament[width=13cm]{89}}
{\small
	\begin{itemize}
		\item Unless otherwise stated, all answers must be mathematically justified.
		\item Partial answers will be graded. 
		\item You can work in groups but each student must write his/her own solution based on his/her own understanding of the problem. Please list on your submission the students you work with for the homework (this will not affect your grade).
		\item Problems with a $(\star)$ are extra credit, they will not (directly) contribute to your score of this homework. However, for every $4$ extra credit questions successfully answered your lowest homework score get replaced by a perfect score.
		\item If you have any questions, feel free to contact me (\texttt{lm4271@nyu.edu}) or to stop at the office hours.
	\end{itemize}
}
\vspace{-0.4cm}
\centerline{\pgfornament[width=13cm]{89}}
\vspace{0.5cm}


%DRAFT

%\color{green}

\begin{problem}[2 points]
	For any two matrices $A,B \in \R^{n\times m}$ we define
	$$
	\langle A,B \rangle_{F} = \Tr(A^{\sT} B).
	$$
	\begin{enumerate}[label=\normalfont(\textbf{\alph*})]
	\item Show that $\langle \cdot, \cdot \rangle_F$ is an inner-product on $\R^{n \times m}$, i.e.\ that it verifies the points of the definition~2.1 of Lecture~4. $\langle \cdot, \cdot \rangle_F$ is called the \emph{Frobenius inner-product}.
		\item The induced norm $\|A\|_F = \sqrt{\Tr(A^{\sT} A)}$ is called the \emph{Frobenius norm}.
			Show that
			$$
			\|A\|_F = \sqrt{\sum_{i=1}^{\min(n,m)} \sigma_i^2},
			$$
			where $\sigma_1, \dots, \sigma_{\min(n,m)}$ are the singular values of $A$.
	\end{enumerate}
\end{problem}

\vspace{1mm}

\begin{problem}[2 points]
	Let $A$ be a $n \times n$ matrix.
	\begin{enumerate}[label=\normalfont(\textbf{\alph*})]
		\item Show that $A$ is invertible if and only if all the singular values of $A$ are non-zero.
		\item We assume here that $A$ is invertible. Show that 
				$$
				\sigma_1(A) \sigma_1(A^{-1}) \geq 1,
				$$
				where $\sigma_1(A)$ and $\sigma_1(A^{-1})$ denote the largest singular value of respectively $A$ and $A^{-1}$.
	\end{enumerate}
\end{problem}

\vspace{1mm}

\begin{problem}[2 points]
	Let $A \in \R^{n\times n}$ be the adjacency matrix of a graph $G$.
	We define a << path from a node $i_1$ to a node $i_k$ >> as a succession of nodes $i_1,i_2, \dots, i_{k}$ such that 
	$$
	i_1 \sim i_2 \sim \cdots \sim i_{k-1} \sim i_k,
	\quad \text{i.e.} \quad
	A_{i_1,i_2} = A_{i_2, i_3} = \cdots = A_{i_{k-1},i_k} = 1.
	$$
	The nodes $i_j$ of the path do not need to be distinct. We say that the path $i_1, \dots, i_k$ has length $k-1$ which is the number of edges in this path.
	The goal of this exercise is to prove that for all $k \geq 1$
	$$
	\mathcal{H}(k): \text{<< For all $i,j \in \{1,\dots,n\}$, the number of paths of length $k$ from $i$ to $j$ is $(A^k)_{i,j}$ >>.} 
	$$
	We will prove that $\mathcal{H}(k)$ holds for all $k$ by induction, that is, we will first prove that $\mathcal{H}(1)$ is true. Then we will prove that if $\mathcal{H}(k)$ is true for some $k$, then $\mathcal{H}(k+1)$ is true. Combining these two things, we get that $\mathcal{H}(2)$ holds, hence $\mathcal{H}(3)$ holds, hence $\mathcal{H}(4)$ holds... and therefore $\mathcal{H}(k)$ will be true for all $k \geq 1$.
	
	\begin{enumerate}[label=\normalfont(\textbf{\alph*})]
		\item Show that $\mathcal{H}(1)$ is true.
		\item Show that if $\mathcal{H}(k)$ is true for some $k$, then $\mathcal{H}(k+1)$ is also true.
	\end{enumerate}


\end{problem}

\vspace{1mm}


\begin{problem}[4 points]
	The goal of this problem is to use spectral clustering techniques on real data.
	The file \texttt{adjacency.tex} contains the adjacency matrix of a graph taken from a social network. This graphs has $n=328$ nodes (that corresponds to users). An edge between user $i$ and user $j$ means that $i$ and $j$ are ``friends'' in the social network.
	The notebook \texttt{friends.ipynb} contains functions to read the adjacency matrix as well as instructions/questions.
	\\

	While we focused in the lectures (and in the notes) on the graph Laplacian
	$$
	L = D - A,
	$$
	where $A$ is the adjacency matrix of the graph, and $D = \Diag(\deg(1), \dots, \deg(n))$ is the degree matrix, we will use here the ``normalized Laplacian'' (instead of $L$)
	$$
	L_{\rm norm} = D^{-1/2} L D^{-1/2} = \Id_n - D^{-1/2} A D^{-1/2},
	$$
	where $D^{-1/2} = \Diag(\deg(1)^{-1/2}, \dots, \deg(n)^{-1/2})$. The reason for using a different Laplacian is that then ``unnormalized Laplacian'' $L$ does not perform well when the degrees in the graph are very broadly distributed, i.e.\ very heterogeneous. In such situations, the normalized Laplacian $L_{\rm norm}$ is supposed to lead to a more consistent clustering.

	\textbf{It is intended that you code in Python and use the provided Jupyter Notebook. Please only submit a pdf version of your notebook (right-click $\to$ `print' $\to$ `Save as pdf').}
\end{problem}


\vspace{1mm}

%\begin{problem}[3 points]
%\end{problem}

%\vspace{1mm}


\begin{problem}[$\star$]
	Define, for $M \in \R^{n \times m}$
	$$
	\|M\|_{\star} = \sum_{i=1}^{\min(n,m)} \sigma_i,
	$$
	where the $\sigma_i$'s are the singular values of $M$. Show that $\| \cdot \|_{\star}$ is a norm (i.e.\ verifies the definition~1.1 of Lecture~4) on $\R^{n \times m}$.
\end{problem}
\vspace{1cm}
\centerline{\pgfornament[width=7cm]{87}}

%\bibliographystyle{plain}
%\bibliography{./references.bib}
\end{document}
