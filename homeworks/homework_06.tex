\documentclass[11pt,nocut]{article}

\usepackage{../latex_style/packages}
\usepackage{../latex_style/notations}

\title{\vspace{-2.0cm}%
	Optimization and Computational Linear Algebra for Data Science\\
Homework 6: Eigenvectors and Markov chains}
%\author{Léo \textsc{Miolane} \ $\cdot$ \ \texttt{leo.miolane@gmail.com}}
\date{\vspace{-1cm}Due on October 15, 2019}
\setcounter{section}{6}

\begin{document}
\maketitle
%\noindent\textbf{Rules:}
\centerline{\pgfornament[width=13cm]{89}}
{\small
	\begin{itemize}
		\item Unless otherwise stated, all answers must be mathematically justified.
		\item Partial answers will be graded. 
		\item You can work in groups but each student must write his/her own solution based on his/her own understanding of the problem. Please list on your submission the students you work with for the homework (this will not affect your grade).
		\item Problems with a $(\star)$ are extra credit, they will not (directly) contribute to your score of this homework. However, for every $4$ extra credit questions successfully answered your lowest homework score get replaced by a perfect score.
		\item If you have any questions, feel free to contact me (\texttt{lm4271@nyu.edu}) or to stop at the office hours.
	\end{itemize}
}
\vspace{-0.4cm}
\centerline{\pgfornament[width=13cm]{89}}
\vspace{0.5cm}


DRAFT

\color{green}

\begin{problem}[4 points]
	The CSV file \texttt{nba17-18.csv} contains the scores of all the games of the NBA baskeball 2017-2018 season, playoffs/finals included. 
	The \emph{Jupyter notebook} \texttt{nba.ipynb} contains functions to read the CSV file and construct the $30 \times 30$ (there is $30$ teams) matrix \texttt{winLoss} where
	$$
	\texttt{winLoss}_{i,j} = \text{number of wins of team $i$ against team $j$.}
	$$
	Using this information, the NBA teams are ranked in \texttt{nba.ipynb} according to their win-loss percentage.
	\begin{enumerate}[label=\normalfont(\textbf{\alph*})]
		\item \label{item:a} Rank the NBA teams according the ``PageRank inspired ranking'' described in Section 4.2 of the lecture notes \#6.
		\item The ranking of question \ref{item:a} uses only the number of win and loss, and not the scores of the games. Propose an evolution of the PageRank method of question \ref{item:a} in order to takes the scores into account, and compute the corresponding ranking.
	\end{enumerate}

	\textbf{It is intented that you use here Python and the provided Jupyter Notebook. Your answers to this problem have to be submitted inside the provided Jupyter notebook. Please make use of comments/markdown cells (where you can type in Latex) in order to make your code/answers readable.}
\end{problem}


\vspace{1mm}


\begin{problem}[2 points]
	Let $A \in \R^{n \times n}$.
	\begin{enumerate}[label=\normalfont(\textbf{\alph*})]
		\item Show that $A^{\sT} A$ positive semi-definite.
		\item Let $M$ be a $n \times n$ symmetric positive semi-definite matrix. Show that there exists $A \in \R^{n \times n}$ such that $M = A^{\sT} A$.
	\end{enumerate}
\end{problem}

\vspace{1mm}

\begin{problem}[3 points]
\end{problem}

\vspace{1mm}


\begin{problem}[$\star$]
\end{problem}
\vspace{1cm}
\centerline{\pgfornament[width=7cm]{87}}

%\bibliographystyle{plain}
%\bibliography{./references.bib}
\end{document}
