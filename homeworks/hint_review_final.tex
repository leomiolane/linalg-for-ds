\documentclass[11pt,nocut]{article}

\usepackage{../latex_style/packages}
\usepackage{../latex_style/notations}

\title{\vspace{-2.0cm}%
	Optimization and Computational Linear Algebra for Data Science\\
Hints for the review exercises}
%\author{Léo \textsc{Miolane} \ $\cdot$ \ \texttt{leo.miolane@gmail.com}}
\date{}
%\setcounter{section}{*}

\begin{document}
\maketitle
%%\noindent\textbf{Rules:}
\centerline{\pgfornament[width=13cm]{89}}
{\small
	\begin{itemize}
		\item Unless otherwise stated, all answers must be mathematically justified.
		\item Partial answers will be graded. 
		\item You can work in groups but each student must write his/her own solution based on his/her own understanding of the problem. Please list on your submission the students you work with for the homework (this will not affect your grade).
		\item Problems with a $(\star)$ are extra credit, they will not (directly) contribute to your score of this homework. However, for every $4$ extra credit questions successfully answered your lowest homework score get replaced by a perfect score.
		\item If you have any questions, feel free to contact me (\texttt{lm4271@nyu.edu}) or to stop at the office hours.
	\end{itemize}
}
\vspace{-0.4cm}
\centerline{\pgfornament[width=13cm]{89}}
\vspace{0.5cm}

\section{Last year's review exercises}

\begin{enumerate}
	\item Show that for all $x \in \R^n$, $ABx=BAx$. (You can decompose such $x$ in the given basis)
	\item (a) See homework 10. (b) Use (a).
	\item Use the definition of eigenvectors/eigenvalues.
	\item Using the spectral Theorem there exists an orthonormal basis $(v_1, \dots, v_n)$ of $\R^n$ consisting of eigenvectors of $A$. Decompose $x$ in such a basis and compute $Ax$.
	\item If $f: \R^2 \to \R$ is convex, and if $(\alpha^*,\beta^*)$ is a minimizer of $f$ then $\nabla f(\alpha^*,\beta^*) = 0$.
	\item Use the definition of $\|x\|_{\infty}$ and $\|x\|$.
	\item By the spectral theorem, you can decompose $x$ in an orthonormal basis of $\R^n$ made of eigenvectors of $A$.
	\item Many possible ways to do this. (a) Show that $\Ker(A^{\sT}) = \Ker(AA^{\sT})$, and then use the rank-nullity theorem and the fact that $\rank(A) = \rank(A^{\sT})$. (b) Compute $AA^{\sT}$ using the SVD of $A$: $A = U \Sigma V^{\sT}$.
	\item (a) Use Lagrange multipliers. (b)  The set of solution of $Ax=b$ is $A^+ b + \Ker(A)$. The result follow from the same arguments than problem 1 of homework 10. 
	\item False.
	\item Show that $\|x+y\|^2 = \|x\|^2 + 2 \langle x,y \rangle + \|y\|^2$.
	\item Show that $\sum_{i=1}^n \langle x,u_i \rangle^2 = \|x\|^2$.
	\item Compute $AA^{\sT}$.
	\item Show that if $\lambda$ is an eigenvalue of $A$ associated with the eigenvector $u$ if and only if $Qu$ is an eigenvector of $B$ with eigenvalue $\lambda$.
	\item Justify that $x = \sum_{i=1}^m \langle x,v_i\rangle v_i$. Then expand $\| \sum \langle x,v_i \rangle v_i \|^2$ and make simplifications.
	\item Use the SVD of $A$.
	\item (a) See Homework 3. (b) Let $V \in \R^{n \times n}$ be the matrix whose columns are $v_1, \dots, v_n$. Show that $\Tr(V^T A v) = \sum_{i=1}^n v_i^{\sT} A v_i$. Then use (a). (c) Use the spectral theorem and (a).
	\item Use Problem 1.b from homework 7.
	\item Expand the right-hand side.
	\item (a). $A^2 = 0$. (b) Take for instance
		$$
		\begin{pmatrix}
			0 & 1 \\
			0 & 0
		\end{pmatrix}
		$$
	\item (a) convex, (b) not convex (c) not convex (d) convex. One can verify these points by computing the Hessian.
	\item (a) ... (b) There is a unique global minima.
	\item $V$ is of dimension $2$, hence $\dim(V^{\perp}) = 4-2 = 2$. $v_1 = (1,-1,0,0)$ and $v_2 = (0, 0, 1,-1)$ work.
	\item (a) Yes. (b) The derivative of a sum is equal to the sum of the derivaties, and the derivatives of $\lambda p$ (for some $\lamdba \in \R$) is equal to $\lambda p'$.
		(c) $\Ker(\mathcal{D})$ is the set of polynomials $p$ that are constant (i.e.\ there exists $a\in\R$ such that $p(x) =a$ for all $x \in \R$).
		(d) $\Im(\mathcal{D}) = \mathcal{P}_{d-1}$.
		(e) (i) check the usual conditions (ii) For polynomial of degree $\leq d$, Taylor formula of order $d$ is exact:
		$$
		T_s(p)(x) = p(x+s) = \sum_{k=0}^d \frac{p^{(k)}(x)}{k!}s^k = \sum_{k=0}^d \frac{\mathcal{D}^k(p)(x)}{k!}s^k.
		$$
	(iii) The matrix has $0$ below the diagonal and for $j \geq i$, $M_{i,j} = \binom{j-1}{i-1}$.
\item (a) Let $B$ be a rank $1$ matrix. One can therefore write $B = u v^T$ for some $u \in \R^m$ and $v \in \R^n$.
	$$
	\|A-B\|_F^2 = \|A\|^2_F - 2 u^T A v + \|u\|^2 \|v\|^2.
	$$
	Now, $u^T Av \leq \|u\| \|v\| \sigma_1$. Hence, writing $r=\|u\|\|v\|$
	$$
	\|A-B\|_F^2 \geq \sum_{i=1}^{\min(n,m)} \sigma_i^2 - 2 \sigma_1 r + r^2
	= \sum_{i=2}^{\min(n,m)} \sigma_i^2 + (\sigma_1 -r)^2
	= \|A-A'\|_{F}^2 + (\sigma_1 - r)^2
	$$
	(b) Let $B = u v^T$ be a rank 1 matrix. Let $v_1, v_2$ be the first two right-singular vectors of $A$. $Span(v)^{\perp}$ has dimension $n-1$, hence one can find a vector of unit norm $z$ in $Span(v)^{\perp} \cap \Span(v_1,v_2)$. We write $z = \alpha_1 v_1 + \alpha_2 v_2$. Since $\|z\|=1$ and $v_1, v_2$ orthogonal, we have $\alpha_1^2 + \alpha_2^2 = 1$. By definition of the spectral norm
	$$
	\|A-B\|_{\rm Sp} \geq \|(A-B)z\| = \|Az\| = \sqrt{\alpha_1^2 \sigma_1^2 + \alpha_2^2 \sigma_2^2} \geq \sigma_2 =  \|A-A'\|_{\rm Sp}.
	$$
\item $xx^T$ is rank $1$ and has two distinct eigenvalues $0$ and $1$. Hence $H$ has two distinct eigenvalues $-1$ and $1$.
\item The vector $(1,1, \dots,1)$ is an eigenvector associated with the eigenvalue $d$. By contradiction let $x$ be an eigenvector associated with the eigenvalue $\lambda >d$. Let $i$ such that $|x_i| = \|x\|_{\infty} >0$. Then
	$$
	|x_i| \lambda 
	= |\sum_{j=1}^n G_{i,j} x_j|
	\leq \sum_{j=1}^n G_{i,j} |x_j|
	\leq |x_i| \sum_{j=1}^n G_{i,j} = d |x_i|.
	$$
	We get a contradiction.
\item Use the matrix product formula.
\item (a) convex but not subspace (b) not convex (c) subspace (hence convex)
\item (a) convex but not strictly convex (b) convex but not strictly convex (c) convex but not strictly convex (d) not convex (e) not convex
\item Cauchy-Schwarz.
\item Apply the spectral theorem to $A$.
\end{enumerate}


\vspace{1cm}
\centerline{\pgfornament[width=7cm]{87}}

%\bibliographystyle{plain}
%\bibliography{./references.bib}
\end{document}
