\documentclass{beamer}

\usepackage{../../latex_style/beamerthemeExecushares}
\usepackage{../../latex_style/notations}

\title{Recitation 3}
\author{Carles Domingo}
\date{Fall 2020}


\begin{document}

\frame{\titlepage} 

\setcounter{showProgressBar}{0}
\setcounter{showSlideNumbers}{1}

\begin{frame}[t]
\frametitle{Rank Nullity Theorem}
\begin{theorem}[Rank-Nullity Theorem]
	Let $L: \R^m \to \R^n$ be a linear transformation. Then
	$$
	\rank(L) + \dim(\Ker(L)) = m.
	$$
\end{theorem}
\begin{itemize}
\item Important theorem (check that you can reproduce the proof).\\
\item Other things to keep in mind:
\begin{itemize}
\item $\text{rank}(AB)\leq \min(\text{rank}(A),\text{rank}(B))$
\item For $c_1, c_2, \dots, c_m \in \R^n$, 
\begin{align*}
\text{rank}(c_1, c_2, \dots, c_m) = \text{rank}(\begin{bmatrix} c_1 & c_2 & \dots & c_m \end{bmatrix}) = \text{rank} \left(\begin{bmatrix} c_1 \\ c_2 \\ \dots \\ c_m \end{bmatrix} \right)
\end{align*}
\end{itemize}
\end{itemize}
\end{frame}

\begin{frame}[t]
\frametitle{Typical exercise} 
\begin{align*}
A = 
\begin{bmatrix}
1 & -2 & 3 \\
0 & 1 & 0 \\
3 & -1 & -1 \\
4 & -2 & 0
\end{bmatrix}
\end{align*}
\begin{enumerate}
\item Find a basis of the kernel of $A$.
\item Find the rank of $A$. Did you need to perform additional computations?
\item Find a basis of the image of $A$. Did you need to perform additional computations?
\end{enumerate}
\end{frame}

\begin{frame}[t]
\frametitle{Typical exercise} 
\begin{align*}
A = 
\begin{bmatrix}
1 & -2 & 3 \\
0 & 1 & 0 \\
3 & -1 & -1 \\
4 & -2 & 0
\end{bmatrix}
\end{align*}
\pause
\pause
\end{frame}

\begin{frame}[t]
\frametitle{Questions: Rank-Nullity Theorem}
Let $A\in \R^{l\times h}$ and $B\in \R^{h\times l}$, and $h>l$.\\
Prove or give a counterexample to the following statements.
\begin{enumerate}
\item $\exists A,B$ s.t $AB$ is invertible.
\item $\exists A,B$ s.t.$BA$ is invertible.
\end{enumerate}
\pause
\pause
\pause
\end{frame}

%----Slide 5-----------------------
\begin{frame}[t]
\frametitle{Symmetric Matrices}
\begin{itemize}
\item $A \in \R^{n\times n}$ symmetric if $A_{ij} = A_{ji}$ for all $i,j \in [1:n]$.
\item Symmetric matrices appear often and have good properties:
\begin{itemize}
\item Orthogonal Projections (Lec. 4) are symmetric.\\
\item Spectral Theorem (Lec. 7) ``symmetric matrices have an orthonormal basis of eigenvectors".\\
\item PCA (Lec. 7): Covariance matrix is symmetric.
\item Convexity (Lec. 9,11): Hessian Matrix (matrix of second derivative) is symmetric
\end{itemize}
\end{itemize}
\end{frame}

\begin{frame}[t]
\frametitle{Questions: Symmetric Matrices}
Let $A\in \R^{k \times n}$. Prove/answer the following statements.
\begin{enumerate}
\item Show that $\forall x\in\R^n$, $ x^\top A^\top Ax\geq0$.
\item When is $x^\top A^\top Ax = 0$ ?
\item Show that $\text{Ker}(A) = \text{Ker}(A^\top A)$.
\item Use this to show $\text{rank}(A)=\text{rank}(A^\top A)$.
\item Now show that $\text{rank}(A)=\text{rank}(A^\top)$.
\end{enumerate}
\pause
\pause
\pause
\pause
\end{frame}

\end{document}
