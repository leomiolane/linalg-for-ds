\documentclass{beamer}
\usepackage{amsthm}
\usepackage[utf8]{inputenc}
\usetheme{Madrid}
\usepackage{../latex_style/packages_old}
\usepackage{../latex_style/notations_old}
\usepackage{outlines}
\usepackage{enumitem}
\renewenvironment{itemize}
\usefonttheme{serif}

\def\labelenumi{\theenumi}
\usefonttheme{serif}

%\setbeamertemplate{itemize items}[default]
%\setbeamertemplate{enumerate items}[default]

% define smaller font command
\newcommand*{\horzbar}{\rule[.5ex]{2.5ex}{0.5pt}}
\newcommand\Fonteight{\fontsize{8}{9.6}\selectfont}
%define enumerate with periods
\renewenvironment{enumerate}%
{\begin{list}{\arabic{enumi}.}% <------ dot here
      {\setlength{\leftmargin}{2.5em}%
       \setlength{\itemsep}{-\parsep}%
       \setlength{\topsep}{-\parskip}%%
       \usecounter{enumi}}%
 }{\end{list}}
%define itemize with arrows
\renewenvironment{itemize}%
{\begin{list}{$\blacktriangleright$}% <------ dot here
      {\setlength{\leftmargin}{2.5em}%
       \setlength{\itemsep}{-\parsep}%
       \setlength{\topsep}{-\parskip}%%
       \usecounter{enumi}}%
 }{\end{list}}

%Information to be included in the title page:
\title{Recitation 3}
\author{Alex Dong}
\institute{CDS, NYU}
\date{Fall 2020}


\makeatletter
\setbeamertemplate{navigation symbols}{}
\setbeamertemplate{footline}{
  \leavevmode%
  \hbox{%
  \begin{beamercolorbox}[wd=.4\paperwidth,ht=2.25ex,dp=1ex,center]{author in head/foot}%
    \usebeamerfont{author in head/foot}\insertshortauthor\expandafter\ifblank\expandafter{\beamer@shortinstitute}{}{~~(\insertshortinstitute)}
  \end{beamercolorbox}%
  \begin{beamercolorbox}[wd=.3\paperwidth,ht=2.25ex,dp=1ex,center]{title in head/foot}%
    \usebeamerfont{title in head/foot}\insertshorttitle
  \end{beamercolorbox}%
  \begin{beamercolorbox}[wd=.3\paperwidth,ht=2.25ex,dp=1ex,right]{date in head/foot}%
    \usebeamerfont{date in head/foot}\insertshortdate{}\hspace*{2em}
    \insertframenumber{} / \inserttotalframenumber\hspace*{2ex} 
  \end{beamercolorbox}}%
  \vskip0pt%
}
\setbeamertemplate{navigation symbols}{}
\makeatother

\begin{document}
%1
\frame{\titlepage} 
%2
\begin{frame}
\frametitle{Rank Nullity Theorem}
\begin{theorem}[Rank-Nullity Theorem (!!!)]
	Let $L: \R^m \to \R^n$ be a linear transformation. Then
	$$
	\rank(L) + \dim(\Ker(L)) = m.
	$$
\end{theorem}
\begin{itemize}
\item \textbf{One of the most important theorems in linear algebra}.\\
\item You should be able to state and prove this theorem (with no notes).
\item The `conservation of dimension' theorem
\begin{itemize}
\item The main rank inequality:
\item $rank(AB)\leq min(rank(A),rank(B))$
\end{itemize}
\item If $n>m$, then the $dim(Im(L)) \leq m$

\end{itemize}
\end{frame}

\begin{frame}
\frametitle{Questions: Rank-Nullity Theorem}
Let $A\in \R^{l\times h}$ and $B\in \R^{h\times l}$, and $h>l$.\\
Prove or give a counterexample to the following statements.
\begin{enumerate}
\item $\exists A,B$ s.t $AB$ is invertible.
\item $\exists A,B$ s.t.$BA$ is invertible.
\end{enumerate}

\end{frame}

\begin{frame}
\frametitle{Questions: Rank-Nullity Theorem}
Let $A\in \R^{\ell\times h}$ and $B\in \R^{h\times \ell}$, and $h>\ell$.\\
Prove or give a counterexample to the following statements.
\begin{solution}
\begin{enumerate}
\item $\exists A,B$ s.t $AB$ is invertible.
 \textbf{True}\\
Consider 
A = $\begin{bmatrix}
1 & 0 & 0 \\
0 & 1 & 0 \\
\end{bmatrix}$, $
B = \begin{bmatrix}
1 & 0 \\
0 & 1 \\
0 & 0 \\
\end{bmatrix}$. 
$AB= \begin{bmatrix}
1 & 0 \\
0 & 1 \\
\end{bmatrix}$ 
\medskip
\item $\exists A,B$ s.t.$BA$ is invertible.
\textbf{False}.\\
In order for $BA$ to be invertible, $rank(BA)=h$. However, $rank(BA) \leq rank(B)\leq \ell $.
\end{enumerate}
\end{solution}
\end{frame}


\begin{frame}
\frametitle{Symmetric Matrices: That's cute!}
\begin{itemize}

\item Symmetric Matrices are not just ``cute"...\\
	\begin{itemize}
	\item They are actually DEEPLY LINKED to many topics in linear algebra.\\
	\end{itemize}
\item Concepts involving Symmetric Matrices
\begin{itemize}
\item Orthogonal Projections (Lec 4) are symmetric.\\
\item Spectral Theorem (Lec 7) ``eigenvectors of symmetric matrices are orthogonal".\\
\item PCA: Covariance matrix is symmetric
\item Concavity: Hessian Matrix (matrix of second derivative) is symmetric
\end{itemize}
\item But, we will see most of this later. For now, just trust me!
\end{itemize}
\end{frame}

\begin{frame}
\frametitle{Questions: Symmetric Matrices}
Let $A\in \R^{k \times n}.$\\
Prove/answer the following statements.
\begin{enumerate}
\item Show that $\forall x\in\R^n$, $ x^TA^TAx\geq0$
\item When is $x^TA^TAx = 0?$
\item Show that $Ker(A) = Ker(A^TA)$
\item Use this to show $rank(A)=rank(A^TA)$
\item Now, show that $rank(A) = rank(A^T)$
\end{enumerate}
\end{frame}


\begin{frame}
\frametitle{Solutions: Symmetric Matrices}

Let $A\in \R^{k \times n}.$\\
\begin{solution}
\begin{enumerate}
\item Show that $\forall x\in\R^n$, $ x^TA^TAx\geq0$ \\
Let $y = Ax$, with y = $\begin{bmatrix} y_1 & \hdots & y_n \\ \end{bmatrix}^T$ \\
Then $x^TA^TAx = (Ax)^T(Ax)=y^Ty = \sum_{i=1}^n {y_i}^2$. \\
Since $y_i$ is a real number, $\sum_{i=1}^n {y_i}^2 \geq 0$.
\medskip
\item What happens when $x^TA^TAx = 0?$ \\
$\sum_{i=1}^n {y_i}^2 = 0 \iff y_i = 0 \quad \forall i\in\{1,...,n\}$ \\
So, $x^TA^TAx = 0 \iff x\in Ker(A)$

\end{enumerate}
\end{solution}
\end{frame}

\begin{frame}
\frametitle{Solutions: Symmetric Matrices}

Let $A\in \R^{k \times n}.$\\
\begin{enumerate}
\item[3.] Show that $Ker(A) = Ker(A^TA)$\\
\end{enumerate}

\begin{solution}
$Ker(A)\subset Ker(A^TA)$ is trivial.\\
\medskip
We now show $Ker(A^TA) \subset Ker(A).$\\
Let $x \in Ker(A^TA)$. \\
Then $A^TAx = 0$.\\
Then $x^T(A^TAx) =x(0) =0 $ \\
By the previous question, then $x\in Ker(A)$.
\end{solution}
\end{frame}

\begin{frame}
\frametitle{Solutions: Symmetric Matrices}

Let $A\in \R^{k \times n}.$\\
Prove or give a counter example to the following statements.
\begin{enumerate}
\item[4.] Use this to show $rank(A)=rank(A^TA)$\\
\end{enumerate}

\begin{solution}
By the rank nullity theorem, \\
\qquad
$n = dim(Ker(A))+rank(A)$ \\
Now, $A^TA \in \R ^{n\times n}$. So by the rank nullity theorem,\\
\qquad $n = dim(Ker(A^TA))+rank(A^TA)$ \\
Setting these equations equal to each other yields: \\
\qquad $dim(Ker(A))+rank(A) = dim(Ker(A^TA))+rank(A^TA)$ \\
And since $Ker(A) = Ker(A^TA)$, then \\
\qquad $rank(A)=rank(A^TA)$
\end{solution}
\end{frame}

\begin{frame}
\frametitle{Solutions: Symmetric Matrices}

Let $A\in \R^{k \times n}.$\\
\begin{enumerate}
\item[5.] Now show $rank(A)=rank(A^T)$\\
\end{enumerate}
\begin{solution}
By the previous question,\\
\qquad $rank(A)=rank(A^TA).\qquad \qquad \qquad \qquad (1)$ \\
%and similarly, \\
%\qquad $rank(A^T)=rank(AA^T)$.\\
Recall that $rank(T_1T_2)\leq min(rank(T_1),rank(T_2))$, and therefore, \\
\qquad $rank(T_1T_2)\leq rank(T_1)$.\\
Applying this to $rank(A^TA)$ yields \\ %and $rank(AA^T)$ yields: \\
\qquad $rank(A^T) \geq rank(A^TA)$ \\ %\quad and \quad $rank(A) \geq rank(AA^T)$.  \\
Replacing $rank(A^TA)$ with $rank(A)$ gives: \\ %and $rank(AA^T)$, we get that\\
\qquad $rank(A^T) \geq rank(A)$ \\ % \quad and \quad $rank(A) \geq rank(A^T)$.  \\
Apply this again for $rank(A^T) = rank(AA^T)$ (start from $(1)$) gives: \\
\qquad $rank(A) \geq rank(A^T)$ \\
So $rank(A)=rank(A^T)$
\end{solution}
\end{frame}



\begin{frame}
\frametitle{Solutions: Matrix Products}
Let $x,y \in \R^{n \times 1}$.
\begin{enumerate}
\item What is the shape and rank of $x^Ty$?
\item What is the shape and rank of $xy^T$?
\item Let $A\in \R^{m\times k}$ and $B \in \R^{k\times n}$. Show that the matrix product AB can be expressed as: $AB=C_1 + \cdots + C_k$ s.t $rank(C_i) \leq 1,$  $\forall i \in \{1,...,k\}$.\\
(Hint, use 2, and try manually calculating for small values of $m,k,n$)
\end{enumerate}
\end{frame}

\begin{frame}
\frametitle{Questions: Matrix Products}
Let $x,y \in \R^{n \times 1}$ both have rank 1.
\begin{solution}
\begin{enumerate}
\item What is the shape and rank of $x^Ty$?\\
Shape is $1\times 1$ and rank is 1 (or 0).
\item What is the shape and rank of $xy^T$?\\
Shape is $n\times n$ and rank is 1 (or 0).
\item Let $A\in \R^{m\times k}$ and $B \in \R^{k\times n}$. Show that the matrix product AB can be expressed as: $AB=C_1 + \cdots + C_k$ s.t $rank(C_i) \leq 1,$  $\forall i \in \{1,...,k\}$.\\
(Hint, use 2, and manually calculating for small values of $m,k,n$)\\
Let $A = \begin{bmatrix}
			\vline & \hdots & \vline  \\
			a_1    & \hdots & a_k  \\
			\vline & \hdots & \vline   \\
		 \end{bmatrix}$
and $B = \begin{bmatrix}
			\horzbar    & b_1    & \horzbar  \\
			\vdots    & \vdots & \vdots  \\
			\horzbar    & b_k    & \horzbar   \\
		 \end{bmatrix}$
\end{enumerate}
\end{solution}
\end{frame}


\end{document}