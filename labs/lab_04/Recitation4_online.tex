\documentclass{beamer}

\usepackage{../../latex_style/beamerthemeExecushares}
\usepackage{../../latex_style/notations}

\title{Recitation 4}
\author{Carles Domingo}
\date{Fall 2020}


\begin{document}

\frame{\titlepage} 

\setcounter{showProgressBar}{0}
\setcounter{showSlideNumbers}{1}

\begin{frame}[t]
\frametitle{Norms}
\begin{definition}[Norm]
	A norm $\| \cdot \|$ on $V$ is a function $\| \cdot \| : V \rightarrow \mathbb{R}_{\geq 0}$ that verifies
	\begin{enumerate}
		\item \emph{Positive definiteness}: if $\|v\| = 0 \implies v=0$. \hfill
		\item \emph{Homogeneity}: $\| \alpha v \| = |\alpha|\times \| v\|$ \hfill 
		\item \emph{Triangle inequality}: $\|u + v\| \leq \|u\| + \|v\|$ \hfill 
	\end{enumerate}
\end{definition}
A norm defines a distance $d(x,y) = \|x-y\|$ on the vector space $V$.
\begin{definition}[Distance]
	A distance $d$ on a set $S$ (not necessarily a vector space) is a function $d : S \times S \rightarrow \R$.
	\begin{enumerate}
		\item \emph{Positive definiteness}: if $d(x,y) \geq 0$ for all $x,y \in S$ and $d(x,y) \geq 0$ iff $x = y$. \hfill
		\item \emph{Symmetry}: $d(x,y) = d(y,x)$ for all $x, y \in S$. \hfill 
		\item \emph{Triangle inequality}: $d(x,y) \leq d(x,z) + d(z,y)$ \hfill 
	\end{enumerate}
\end{definition}
\end{frame}

\begin{frame}[t]
\frametitle{Inner Products}
\begin{definition}[Inner product]
	Let $V$ be a vector space.
	An inner product on $V$ is a function $\langle \cdot, \cdot \rangle$ from $V \times V$ to $\R$ that verifies the following points:
	\begin{enumerate}
		\item \emph{Symmetry}: $\langle u, v \rangle = \langle v, u\rangle$ for all $u,v \in V$.
		\item \emph{Linearity}: $\langle u+v, w \rangle = \langle u, w\rangle + \langle v, w\rangle$ and $\langle \alpha v, w \rangle = \alpha \langle v, w \rangle$ for all $u,v,w \in V$ and $\alpha \in \R$.
		\item \emph{Positive definiteness}: $\langle v, v\rangle \geq 0$ with equality if and only if $v = 0$.
\end{enumerate}
\end{definition}
\begin{itemize}
\item Important: An inner product defines a norm $\|u\| = \sqrt{\langle u, u \rangle}$.
\item Most used inner product: the Euclidean inner product: $\langle u,v \rangle = u^\top v$. The corresponding norm is $\|u\|= \sqrt{u^\top u} = \sqrt{}\sum_{i=1}^n u_i^2$.
\item Given an inner product, we can define the angle between two vectors: $cos(\theta) = \frac{ \langle u,v \rangle }{\|u\| \|v\|}$.
\end{itemize}
\end{frame}

\begin{frame}[t]
\frametitle{Questions: Norms \& Inner Products}
1. Which of the following functions are inner products for $x,y\in\R^3$?
\begin{enumerate}
\item[i.] $f(x,y) = x_1y_2+x_2y_3+x_3y_1$
\item[ii.] $f(x,y) = x_1^2y_1^2+x_2^2y_2^2+x_1^2y_1^2$
\item[iii.] $f(x,y) = x_1y_1+x_3y_3$
\end{enumerate}
\pause
\pause
\end{frame}

\begin{frame}[t]
\frametitle{Questions: Norms \& Inner Products}
2. For $A \in \R^{m\times n}$ and $x\in \R^n$, prove that
$$ \|Ax\| \leq \|x\|\sqrt{\sum_{i=1}^m \sum_{j=1}^n} A_{i,j}^2$$
\pause
\end{frame}

\begin{frame}[t]
\frametitle{Questions: Orthogonality}
Recall from the lecture:
\begin{enumerate}
\item Two vectors $u,v$ are orthogonal if $\langle u, v \rangle = 0$.
\item If $U$ is a subspace of of a vector space $V$ with inner product $\langle , \rangle$, the orthogonal projection $P_{U} : V \rightarrow U$ is defined as $x \mapsto \argmin_{u \in U} \|x-u\|$.
\end{enumerate}
Exercises:
\begin{enumerate}
\item Let $v_1,...,v_k$ be a list of orthogonal vectors. Show that $v_1,...,v_k$ are linearly independent.
\item Let $U$ be the subspace of $\R^n$ with orthonormal basis $u_1,...,u_k$.
\begin{enumerate}
 \item[i.] Prove that the orthogonal projection of $v \in \R^n$ onto $U$ can be expressed as $P_U  = \sum_{i=0}^k  \langle v,u_i \rangle u_i$. (Use the fact that the orthonormal basis for a subspace of $\R$ can be extended to obtain an orthonormal basis for $\R$).
 \item[ii.] Prove that $P_U(v)\leq \|v\|$.
 \item[iii.] Prove that $v-P_U(v)$ is orthogonal to $P_U(v)$.
\end{enumerate}
\end{enumerate}
\end{frame}

\begin{frame}[t]
\frametitle{Questions: Orthogonality}
1. Let $v_1,...,v_k$ be a list of orthogonal vectors. Show that $v_1,...,v_k$ are linearly independent.
\pause
\end{frame}

\begin{frame}[t]
\frametitle{Questions: Orthogonality}
2. Let $U$ be the subspace of $\R^n$ with orthonormal basis $u_1,...,u_k$.
\begin{enumerate}
 \item[i.] Prove that the orthogonal projection of $v \in \R^n$ onto $U$ can be expressed as $P_U  = \sum_{i=0}^k  \langle v,u_i \rangle u_i$. (Use the fact that the orthonormal basis for a subspace of $\R$ can be extended to obtain an orthonormal basis for $\R$).
 \item[ii.] Prove that $P_U(v)\leq \|v\|$.
 \item[iii.] Prove that $v-P_U(v)$ is orthogonal to $P_U(v)$.
 \end{enumerate}
\pause
\pause
\pause
\pause
\end{frame}

\begin{frame}[t]
\frametitle{Questions: Orthogonal Complements}
\begin{enumerate}
\item Let $S, U$ be subspaces of a vector space $V$.\\
Prove the following statement:
$S\subset U \implies S^\perp \supset U^\perp$
\bigskip
\item[2.] Let $A\in\R^{n\times m}$.
Assume the Euclidean inner product. Prove that $Im(A^T) = Ker(A)^\perp$.

(Hint: $\implies$ is easy. Use (1) for $\impliedby$)
\end{enumerate}
\end{frame}

\begin{frame}[t]
\frametitle{Questions: Orthogonal Complements}
1. Let $S, U$ be subspaces of a vector space $V$.\\
Prove the following statement:
$S\subset U \implies S^\perp \supset U^\perp$
\end{frame}

\begin{frame}[t]
\frametitle{Questions: Orthogonal Complements}
2. Let $A\in\R^{n\times m}$.
Assume the Euclidean inner product. Prove that $Im(A^T) = Ker(A)^\perp$.

(Hint: $\implies$ is easy. Use (1) for $\impliedby$)
\pause
\end{frame}

\begin{frame}[t]
\frametitle{Questions: Idempotence}
\begin{definition}[Idempotence]
An matrix P is idempotent when $P^2=P$.
\end{definition}
\bigskip
\begin{enumerate}
\item Show that $X(X^TX)^{-1}X^T$ is idempotent.
\item Show that all orthogonal projections are idempotent.
\item Give an example of an idempotent matrix that is not an orthogonal projection.  \\
(Hint: Show that your matrix does not minimize the distance to subspace it projects onto.)
\end{enumerate}
\end{frame}

\begin{frame}[t]
\frametitle{Questions: Idempotence}
\begin{enumerate}
\item Show that $X(X^TX)^{-1}X^T$ is idempotent.
\item Show that all orthogonal projections are idempotent.
\pause
\end{enumerate}
\end{frame}

\begin{frame}[t]
\frametitle{Questions: Idempotence}
3. Give an example of an idempotent matrix that is not an orthogonal projection.  \\
(Hint: Show that your matrix does not minimize the distance to subspace it projects onto.)
\pause
\end{frame}

\end{document}
