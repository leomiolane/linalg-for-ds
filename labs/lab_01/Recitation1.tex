\documentclass{beamer}
\usepackage{amsthm}
\usepackage[utf8]{inputenc}
\usetheme{Madrid}
\usepackage{../latex_style/packages_old}
\usepackage{../latex_style/notations_old}
\usepackage{outlines}
\usepackage{enumitem}
\renewenvironment{itemize}
\usefonttheme{serif}
\def\labelenumi{\theenumi}
\usefonttheme{serif}

%\setbeamertemplate{itemize items}[default]
%\setbeamertemplate{enumerate items}[default]

% define smaller font command
\newcommand*{\horzbar}{\rule[.5ex]{2.5ex}{0.5pt}}
\newcommand\fonteight{\fontsize{8}{9.6}\selectfont}
\newcommand\fontten{\fontsize{10}{1.2}\selectfont}

%define enumerate with periods
\renewenvironment{enumerate}%
{\begin{list}{\arabic{enumi}.}%  \langle ------ dot here
      {\setlength{\leftmargin}{2.5em}%
       \setlength{\itemsep}{-\parsep}%
       \setlength{\topsep}{-\parskip}%%
       \usecounter{enumi}}%
 }{\end{list}}
%define itemize with arrows
\renewenvironment{itemize}%
{\begin{list}{$\blacktriangleright$}%  \langle ------ dot here
      {\setlength{\leftmargin}{2.5em}%
       \setlength{\itemsep}{-\parsep}%
       \setlength{\topsep}{-\parskip}%%
       \usecounter{enumi}}%
 }{\end{list}}

%Information to be included in the title page:
\title{Recitation 1}
\author{Alex Dong}
\institute{CDS, NYU}
\date{Fall 2020}


\makeatletter
\setbeamertemplate{navigation symbols}{}
\setbeamertemplate{footline}{
  \leavevmode%
  \hbox{%
  \begin{beamercolorbox}[wd=.4\paperwidth,ht=2.25ex,dp=1ex,center]{author in head/foot}%
    \usebeamerfont{author in head/foot}\insertshortauthor\expandafter\ifblank\expandafter{\beamer@shortinstitute}{}{~~(\insertshortinstitute)}
  \end{beamercolorbox}%
  \begin{beamercolorbox}[wd=.3\paperwidth,ht=2.25ex,dp=1ex,center]{title in head/foot}%
    \usebeamerfont{title in head/foot}\insertshorttitle
  \end{beamercolorbox}%
  \begin{beamercolorbox}[wd=.3\paperwidth,ht=2.25ex,dp=1ex,right]{date in head/foot}%
    \usebeamerfont{date in head/foot}\insertshortdate{}\hspace*{2em}
    \insertframenumber{} / \inserttotalframenumber\hspace*{2ex} 
  \end{beamercolorbox}}%
  \vskip0pt%
}
\setbeamertemplate{navigation symbols}{}
\makeatother

\begin{document}
%----Slide 1-----------------------
\frame{\titlepage} 
%----Slide 2-----------------------
\begin{frame}
\frametitle{Announcements}
Welcome!
\begin{itemize}
\item My Office hours: TBA (once I figure my schedule out)
\item Rotating cohorts - if you are not scheduled to be in person, please join the online zoom call (w/ Carles or Irina)
\item Recitations are for practice problems
\item You will have time in recitation to solve the problems.
\item Recitations will be released so you can look over/solve problems in advance.
\end{itemize}
\end{frame}

%----Slide 3-----------------------
\begin{frame}
\frametitle{Why is Linear Algebra important?}
Linear Algebra...
\begin{itemize}
\item Appears in ALL applied math, including \textit{data science}
\item Is solvable. If you can write it down, you can solve it! \\
    \qquad (not true for other math, e.g Diff Eq, Integrals)
\item Is \textit{fundamental} to understanding tools in machine learning
\end{itemize}

Relevant applications we will cover in the class:
\begin{itemize}
\item Linear Regression
\item Principal Component Analysis
\item Gradient Descent
\end{itemize}

First year MSDS students are \textit{highly encouraged} to take this class.
\end{frame}
%----Slide 4-----------------------
\begin{frame}
\frametitle{Concept Review: Vector Spaces}

\begin{definition}[Vector space]
	    $V$ is a vector space over field $F$ when
	    \fonteight
		\begin{enumerate}
		\item[1.] Closure under Addition: $x+y \in V$
		\item[2.] Sum is commutative ($x+y = y+x$) and associative $x+(y+z)= (x+y)+z$
		\item[3.] Additive Identity (in $V$): $0\in V$ ($x+0 =x$)
		\item[4.] Additive Inverse: ($\forall x, \exists -x $ s.t $ x+(-x)=0$
		\item[5.] Closure under Scalar Multiplication: $\alpha x \in V$
		\item[6.] Multiplicative Identity (in $F$): $\alpha x \in V$
		\item[7.] Compatibility in Multiplication: $\alpha (\beta x) = (\alpha \beta) x $
		\item[8.] Distributivity: $(\alpha + \beta) x = \alpha x + \beta \cdot \vec{y}
			 \text{ and }
			\alpha  (x + y) = \alpha  x + \alpha \cdot y.$
		\end{enumerate}
\end{definition}
Notice that ``vectors" are not explicitly defined. \\
Optional: prove (after class) that $\R^3$ with standard definitions for addition and scalar multiplication
 is vector space over field $\R$.
\end{frame}

%----Slide 5-----------------------
\begin{frame}
\frametitle{Concept Review: Vector Spaces}
\begin{itemize}
\item In this class,
\begin{itemize}
\item Our field is always $\R$. ($\C$ is also a field.)
\item Standard definitions for vector addition, and scalar multiplication.
\item But, V is (usually) $\R^n$, or (sometimes) $\R^{n\times n}$.
\end{itemize}

\item Everything in linear algebra is in a vector space!
\item (!) A recurring concept in data science is to ``vectorize'' problems
\begin{itemize}
\item If you can transform/reframe your problem in linear algebra, you can (attempt) to solve it!
\end{itemize}
\end{itemize}
\end{frame}

%----Slide 6-----------------------
\begin{frame}
\frametitle{Concept Review: Subspaces}
\begin{definition}[Subspace]
	A subset $S$ of a vector space $V$ is a \emph{subspace} if it is closed under addition and scalar multiplication.
	\begin{enumerate}
		\item[1.] Closure under Addition: $x+y \in V$
		\item[2.] Closure under Scalar Multiplication: $\alpha x \in V$
	\end{enumerate}
\end{definition}
\begin{itemize}
\item A subspace is also a vector space!\\
\item Everything in linear algebra is in a vector space. \\ 
\item Anything \textit{interesting} in linear algebra is in a subspace. \\
\item Subspaces are a recurring concept throughout this entire course. \\
\end{itemize}
\end{frame}

%----Slide 7-----------------------
\begin{frame}
\frametitle{Questions 1: Subspaces, Span}
  Recall that $\R^2=\{(x,y) : x,y\in\R \}$ can be thought
  of as the $xy$-plane.  Consider the two vectors $v=(1,1)$ and
  $w=(-1,2)$.  Describe the following sets geometrically.  Which are
  subspaces of $\R^2$?
  \begin{enumerate}
  \item $\Span(v)$
  \item $\Span(v,w)$
  \item $\Span(v)\cup\Span(w)$, that is, the vectors in $\Span(v)$ or
    $\Span(w)$
  \item $\Span(v)\cap\Span(w)$, that is, the vectors in both $\Span(v)$ and
    $\Span(w)$
  \item $\{(1-t)v + tw : t\in[0,1]\}$
  \item $\{(1-t)v+tw : t\in\R\}$
  \item $\{\alpha v + \beta w : \alpha,\beta\geq 0\}$
  \item $\Span(v,w,x)$ where $x=(0,5)$.
  \item $\{(a,b)\in\R^2 : a^2+b^2\leq 25\}$
  \item $\{(a,a)\in\R^2 : a\in\R\}$
  \item $\{(a,a^2)\in\R^2 : a\in\R\}$
  \item $\{(a,1)\in\R^2 : a\in\R\}$
  \end{enumerate}
\end{frame}

%----Slide 8-----------------------
\begin{frame}

\frametitle{Solutions 1: Subspaces, Span }

 Recall that $\R^2=\{(x,y) : x,y\in\R \}$ can be thought
  of as the $xy$-plane.  Consider the two vectors $v=(1,1)$ and
  $w=(-1,2)$.  Describe the following sets geometrically.  Which are
  subspaces of $\R^2$?
\begin{enumerate}
  \item $\Span(v)$ \hfill True
  \item $\Span(v,w)$ \hfill True
  \item $\Span(v)\cup\Span(w)$, \hfill False
  \item $\Span(v)\cap\Span(w)$, \hfill True
  \item $\{(1-t)v + tw : t\in[0,1]\}$ \hfill False
  \item $\{(1-t)v+tw : t\in\R\}$ \hfill False
  \item $\{\alpha v + \beta w : \alpha,\beta\geq 0\}$ \hfill False
  \item $\Span(v,w,x)$ where $x=(0,5)$. \hfill True
  \item $\{(a,b)\in\R^2 : a^2+b^2\leq 25\}$ \hfill False
  \item $\{(a,a)\in\R^2 : a\in\R\}$ \hfill True
  \item $\{(a,a^2)\in\R^2 : a\in\R\}$ \hfill False
  \item $\{(a,1)\in\R^2 : a\in\R\}$\hfill  False
\end{enumerate}
\end{frame}

%----Slide 9-----------------------
\begin{frame}
\frametitle{Questions 2: Linear Independence, Bases, }

\begin{enumerate}
\item Let $v_1,v_2,v_3,v_4$ (all distinct) $\in \R^3$.\\
Let $C_1 = \{v_1,v_2\}; C_2 = \{v_3,v_4\}$. \\
If $C_1$ and $C_2$ are both linearly independent, 
what are the possible values for $dim(Span\{v_1,v_2,v_3,v_4\})$? (No formal proof necessary)
\smallskip
\item Let $v_1,...,v_n \in \R^n$ be a basis of $\R^n$. \\
Prove that for $x \in R^n$, there exists unique $\alpha_1,...,\alpha_n$
such that $x = \sum_{i=0}^n \alpha_i v_i$.
\smallskip
\item Let $v_1,...,v_m \in \R^n$ be linearly dependent. \\
Prove that for $x \in Span(v_1,...,v_m)$, 
there exist infinitely many $\alpha_1,...,\alpha_m \in \R$ 
such that $x = \sum_{i=0}^m \alpha_i v_i$.
\smallskip
\item True or False: If $B=\{v_1,\ldots,v_n\}$ is a basis for $\R^n$,
  and $W$ is a subspace of $\R^n$, then some subset of $B$ is a basis
  for $W$.
\end{enumerate}
\end{frame}

%----Slide 10-----------------------
\begin{frame}
\frametitle{Solutions 2: Linear Independence, Bases }

\begin{enumerate}
	\item[1.] Let $v_1,v_2,v_3,v_4$ (all distinct) $\in \R^3$ .\\
		Let $C_1 = \{v_1,v_2\}; C_2 = \{v_3,v_4\}$. \\
		If $C_1$ and $C_2$ are both linearly independent, 
		what are the possible values for $dim(Span\{v_1,v_2,v_3,v_4\})$? (No formal proof necessary.)
	\begin{solution}
		Either $C_1\subset Span(C_2)$ and the dimension is $2$, or the dimension is $3$.
	\end{solution}
\end{enumerate}
\end{frame}
%----Slide 11-----------------------

\begin{frame}
\frametitle{Solutions 2: Linear Independence, Bases}
\begin{enumerate}
	\item[2.] Let $v_1,...,v_n \in \R^n$ be a basis of $\R^n$. \\
		Prove that for $x \in R^n$, there exists unique $\alpha_1,...,\alpha_n$
		such that $x = \sum_{i=0}^n \alpha_i v_i$.
	\begin{solution}
	By definition of basis, $v_1,...,v_n$ is a linearly independent set, and spans $\R^n$. Since x $\in \R^n$, 
	\begin{center}
		$\exists \alpha_1,...,\alpha_n \text{ s.t } x = \sum_{i=0}^n \alpha_i v_i.$ 
	\end{center}
	Let $\beta_1,...,\beta_n$ s.t $x = \sum_{i=0}^n \beta_i v_i$. Then,
	\begin{center}
		$x-x=0=\sum_{i=0}^n (\alpha_i-\beta_i) v_i$
	\end{center} 
	Then by definition of linear independence, 
	\begin{center}
		$\alpha_i-\beta_i=0  \qquad \forall i \in 1,...,n$
	\end{center}
	So $\alpha_i=\beta_i, \forall i \in 1,...,n$ 
	\end{solution}
\end{enumerate}
\end{frame}

%----Slide 12-----------------------
\begin{frame}
\frametitle{Solutions 2: Linear Independence, Bases}
\begin{enumerate}
\item[3.] Let $v_1,...,v_m \in \R^n$ be linearly dependent. \\
Prove that for $x \in Span(v_1,...,v_m)$, 
there exist infinitely many $\alpha_1,...,\alpha_m \in \R$ 
such that $x = \sum_{i=0}^m \alpha_i v_i$.
\begin{solution}
    \fonteight
    By assumption, $x\in Span(v_1,\ldots,v_m)$. So
    \begin{center}
    $\exists \beta_1,...,\beta_m$ s.t $x=\sum_{i=1}^m \beta_iv_i$
    \end{center}
    
    Since $v_1,\ldots,v_m$ are linearly dependent, there are $\gamma_1,\ldots,\gamma_m\in \R$ such that
    \begin{center} $\sum_{i=0}^m\gamma_i v_i= 0$ \end{center}
    where not all $\gamma_i=0$.  \\
    Now, let $r \in \R$. Then,
    \begin{center} $x = x+0 = \sum_{i=1}^m \beta_iv_i + r\sum_{i=0}^m\gamma_i v_i
    =  \sum_{i=1}^m (\beta_i +r\gamma_i)v_i $ \end{center} 
    This gives infinitely many distinct $\alpha$ where $\alpha_i=\beta_i+r\gamma_i$ for $r\in\R$.

\end{solution}
\end{enumerate}
\end{frame}

%----Slide 13-----------------------
\begin{frame}
\frametitle{Solutions 2: Linear Independence, Bases}
\begin{enumerate}
 \item[4.] True or False: If $B=\{v_1,\ldots,v_n\}$ is a basis for $\R^n$,
  and $W$ is a subspace of $\R^n$, then some subset of $B$ is a basis
  for $W$.
  
\begin{solution}
False. Consider $B=\{(1,0),(0,1)\}$ and $W=Span((1,1))$.
\end{solution}

\end{enumerate}
\end{frame}

%----Slide 14-----------------------

\begin{frame}
\frametitle{Questions 3: Bases, Dimension}
Let $V$ be the set of functions \\
$$V \coloneqq \{ p: \R \to \R\ |\ p(x) = \sum_{k=0}^n a_k x^k\}$$ \\
where $a_0,\dots,a_n \in \R$, and $x\in \R$ is \textit{constant}.
\medskip
\begin{enumerate}
	\item What kind of function does this set contain?
	\item Define an addition operation $+: V\times V \to V$,\\
	 and a scalar multiplication operation $\cdot: \R \times V \to V$, \\
	 such that the triple $(V,+,\cdot$) is a vector space.
	\item Find a basis for this vector space.
	\item What is the dimension of this vector space?
\end{enumerate}
\end{frame}

%----Slide 15-----------------------
\begin{frame}
\frametitle{Solutions 3: Bases, Dimension}
\begin{enumerate}
	\item What kind of function does this set contain?\\
	\item Define an addition operation $+: V\times V \to V$,\\
	 and a scalar multiplication operation $\cdot: \R \times V \to V$, \\
	 such that the triple $(V,+,\cdot$) is a vector space.\\
	\item Find a basis for this vector space.\\
	\item What is the dimension of this vector space?\\
\end{enumerate}

\begin{solution}
\begin{enumerate}
	\item Polynomials evaluated at x
	\item Standard definitions for function addition and scalar multiplication	 
	\item ${1,x,x^2,...,x^n}$
	\item $n+1$
\end{enumerate}
\end{solution}
\end{frame}

\end{document}