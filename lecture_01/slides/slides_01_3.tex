\documentclass{beamer}

\usepackage{../../latex_style/beamerthemeExecushares}
\usepackage{../../latex_style/notations}

\title{Lecture 1.3: Span, Linear dependency and dimension}
\subtitle{Optimization and Computational Linear Algebra for Data Science}
\author{Léo Miolane}
\date{}

\setcounter{showSlideNumbers}{1}

\begin{document}
\setcounter{showProgressBar}{0}
\setcounter{showSlideNumbers}{0}

\frame{\titlepage}

\begin{frame}
	\frametitle{Contents}
	\begin{enumerate}
		\item Linear combination and span
		\item Linear dependency
		\item Basis and dimension
	\end{enumerate}
\end{frame}


\setcounter{framenumber}{0}
\setcounter{showSlideNumbers}{1}

\section{Linear combination \& Span}

\begin{frame}[t]{Linear combination}
	Let $V$ be a vector space (think for instance $V=\R^n$).
	\begin{block}{\bf Definition}
		We say that $y \in V$ is a \emph{linear combination} of the vectors $x_1, \dots, x_k \in V$ if there exists $\alpha_1, \dots, \alpha_k \in \R$ such that
		$$
		y = \sum_{i=1}^k \alpha_i x_i = \alpha_1 x_1 + \cdots + \alpha_k x_k.
		$$
	\end{block}
\end{frame}


\begin{frame}[t]{Remarks}
	\begin{itemize}
		\item A linear combination is always a finite sum.
		\item If $S$ is a subspace of $V$, then any linear combination of vectors $x_1, \dots, x_k$ of $S$ is also in $S$:
			$$
			\alpha_1 x_1 + \cdots + \alpha_k x_k \in S, \qquad \text{for all  } \alpha_1, \dots, \alpha_k \in \R.
			$$
	\end{itemize}
			\vspace{0.2cm}
			\begin{center}
				<< Subspaces are closed under linear combinations. >>
			\end{center}

			\vspace{1cm}
			\textbf{Exercise:} Prove it !
\end{frame}


\begin{frame}[t]{Span}
	\begin{block}{\bf Definition}
		Let $x_1, \dots, x_k$ be vectors of $V$. We define the \emph{linear span} of $x_1, \dots, x_k$ as the set of all linear combinations of these vectors:
		$$
		\Span(x_1, \dots, x_k) \defeq
		\Big\{ \alpha_1 x_1 + \cdots + \alpha_k x_k \, \Big| \, \alpha_1, \dots, \alpha_k \in \R \Big\}.
		$$
	\end{block}
\end{frame}

\section{Linear dependency}

\begin{frame}[t]{Linear dependency}
	\begin{block}{\bf Definition}
		Vectors $x_1, \dots x_k \in V$ are \emph{linearly dependent} is there exists $\alpha_1, \dots, \alpha_k \in \R$ \textbf{that are not all zero} such that 
		$$
		\alpha_1 x_1 + \cdots + \alpha_k x_k = 0.
		$$
		They are said to be \emph{linearly independent} otherwise.
	\end{block}

	\vspace{0.5cm}

	\textbf{Key observation:}
<< $x_1, \dots, x_k$ are linearly dependent >>
\emph{is equivalent to}
<< one of the vectors $x_1, \dots, x_k$ can be obtained as a linear combination of the others.>>
\end{frame}
\begin{frame}[t]{Why ?}
\end{frame}

\section{Basis, dimension}

\begin{frame}[t]{Basis}
	\begin{block}{\bf Definition}
	A family $(x_1, \dots, x_n)$ of vectors of $V$ is a basis of $V$ if
	\begin{enumerate}
		\item $x_1, \dots, x_n$ are linearly independent,
		\item $\Span(x_1, \dots, x_n) = V$.
	\end{enumerate}
	\end{block}
	\vspace{0.5cm}
	This means that $(x_1, \dots, x_n)$ is a basis of $V$ if
	\begin{enumerate}
		\item None of the $x_i$ is a linear combination of the others $(x_j)_{j \neq i}$.
		\item Any vector of $V$ can be expressed as a linear combination of $(x_1, \dots, x_n)$.
	\end{enumerate}
\end{frame}

\begin{frame}[t]{Example: the canonical basis of $\R^n$}
	Let us define the vectors $e_1, \dots, e_n \in \R^n$ by
	\begin{align*}
		e_1 &= (1, 0, 0, \dots, 0) \\
		e_2 &= (0, 1, 0, \dots, 0) \\
		\vdots & \\
		e_n &= (0, 0, 0, \dots, 1).
	\end{align*}
	One can verify (exercise!) that the family $(e_1, \dots, e_n)$ is a basis of $\R^n$. This basis is called the ``canonical basis'' of $\R^n$. We conclude that $\R^n$ has dimension $n$.
\end{frame}

\begin{frame}[t]{Dimension}
	\begin{block}{\bf Theorem}
	Let $V$ be a vector space.
	\begin{itemize}
		\item If $V$ admits a basis $(v_1, \dots, v_n)$, then every basis of $V$ has also $n$ vectors. We say that $V$ has dimension $n$ and write $\dim(V) = n$.
		\item Otherwise, we say that $V$ has infinite dimension: $\dim(V) = +\infty$.
	\end{itemize}
	\end{block}
	\textbf{Example:} 
	\begin{itemize}
		\item $\R^n$ has dimension $n$, because the canonical basis $(e_1,\dots, e_n)$ is a basis of $\R^n$ with $n$ vectors.
		\item $\big\{ f \, | \, f: \R \to \R \big\}$ has infinite dimension.
	\end{itemize}
\end{frame}


%\appendix
%\backupbegin
%\begin{frame}
%\frametitle{Backup slide 1}
%Hello
%\end{frame}
%\backupend

\end{document}
