\documentclass[11pt,nocut]{article}

\usepackage{../latex_style/packages}
\usepackage{../latex_style/notations}
%\externaldocument{../lecture_02/lecture_02}
\externaldocument{../lecture_07/lecture_07}


\title{\vspace{-2.0cm}%
	Optimization and Computational Linear Algebra for Data Science\\
Lecture 11: Linear regression, matrix completion}
\author{Léo \textsc{Miolane} \ $\cdot$ \ \texttt{leo.miolane@gmail.com}}
\date{\today}

\begin{document}
\maketitle
\textbf{Warning:}
\emph{This material is not meant to be lecture notes. It only gathers the main concepts and results from the lecture, without any additional explanation, motivation, examples, figures...
}


\section{Least squares}

Assume that we are given point $a_i = (a_{i,1}, \dots, a_{i,d}) \in \R^d$ with labels $y_i \in \R$ for $i=1 \dots n$.
We aim at finding a vector $x\in\R^d$ such that
$$
y_i \simeq \langle a_i, x \rangle = \sum_{j=1}^d a_{i,j} x_j, \qquad \text{for} \ \ i=1 \dots n.
$$
If we denote by $A$ the $n \times d$ matrix whose rows are $a_1, \dots, a_n$, i.e.\ $A_{i,j} = a_{i,j}$, we are looking for some $x$ such that $Ax \simeq y$.

\subsection{Solving the system $Ax=y$}

As we have seen in Lecture 2, we can distinguish two cases:
\begin{itemize}
	\item If $y \not\in \Im(A)$ then the equation $Ax=y$ does not admit any solution (by definition of $\Im(A)$).
	\item If $y \in \Im(A)$ then the equation $Ax=y$ admits at least a solution $x_0$ (by definition of $\Im(A)$). Moreover, the set of (all) solutions is
		$$
		x_0 + \Ker(A) = \{x_0 + v \, | \, v \in \Ker(A) \}.
		$$
		In particular, if $\Ker(A) = \{0\}$ then the equation admits a unique solution.
\end{itemize}

In the second case, one can obtain an expression for a particular solution $x_0$ using the SVD of $A$.
Let $r = \rank(A)$, $\sigma_1, \sigma_2, \dots, \sigma_r >0$ be the non-zero singular values of $A$ and $\Sigma = \Diag(\sigma_1, \dots, \sigma_r)$. Finally, let $A = U \Sigma V^{\sT}$ be the SVD of $A$, where $V \in \R^{n \times r}$ and $U \in \R^{d \times r}$ are matrices that have orthonormal columns. 
\\

Notice that $V^{\sT} V = \Id$ and that $U U^{\sT}$ is the orthogonal projection on $\Im(A)$. Hence, if we let $x_0 = V \Sigma^{-1} U^{\sT} y$, we have
$$
A x_0 =U \Sigma V^{\sT}V \Sigma^{-1} U^{\sT} y
= U U^{\sT} y = y
$$
because we assumed that $y \in \Im(A)$.
This motivates the following definition:
\begin{definition}[Moore-Penrose pseudo-inverse]
	The matrix $A^{\dagger} \defeq V \Sigma^{-1} U^{\sT}$ is called the (Moore-Penrose) pseudo-inverse of $A$.
\end{definition}

Notice that in the case where $A$ is invertible, $A^{\dagger} = A^{-1}$.
From the analysis above, we deduce:
\begin{proposition}\label{prop:linear_system}
	The set of solution of the linear system $Ax = y$ is
	\begin{itemize}
		\item $\emptyset$ if $y \not\in \Im(A)$.
		\item $A^{\dagger}y + \Ker(A)$ otherwise.
	\end{itemize}
\end{proposition}

\subsection{Least squares}

In general, there is no reason for $y$ to belong to $\Im(A)$, especially when $n > d$. (Exercise: why?)
Therefore one is rather interested by solving
\begin{equation}\label{eq:least_squares}
\min_{x \in \R^d} \| Ax - y \|^2.
\end{equation}
The function $f: x \mapsto \|Ax - y\|^2$ is convex (Exercise: why?) and differentiable. Hence
$x$ is solution of \eqref{eq:least_squares} if and only if $\nabla f (x) = 0$. Compute
$$
f(x) = (Ax - y)^{\sT}(Ax - y) = x^{\sT} A^{\sT} A x - 2 y^{\sT} A x + \|y\|^2.
$$
Hence $\nabla f(x) = 2 A^{\sT} A x - 2 A^{\sT} y$. We conclude
$$
x \ \ \text{is solution of} \ \ \eqref{eq:least_squares} \qquad
\Longleftrightarrow
\qquad A^{\sT} A x = A^{\sT} y.
$$
If $A^{\sT} A$ is invertible there is a unique minimizer $x^* = (A^{\sT} A)^{-1} A^{\sT} y$.
In the general case, we see that the solutions of \eqref{eq:least_squares} are the solutions of the linear system $A^{\sT} A x = A^{\sT} y$. From Proposition \ref{prop:linear_system} we get that the solutions of \eqref{eq:least_squares} are
$$
(A^{\sT}A)^{\dagger} A^{\sT}y + \Ker(A^{\sT} A).
$$
This expression simplifies a lot. First (exercise!) we have $\Ker(A^{\sT} A) = \Ker(A)$.
Then if we let $A = U \Sigma V^{\sT}$ be the SVD of $A$, we have
$$
A^{\sT} A = V \Sigma^2 V^{\sT}.
$$
$V \Sigma^2 V^{\sT}$ is therefore the SVD of $A^{\sT} A$. Hence $(A^{\sT} A)^{\dagger} = V \Sigma^{-2} V^{\sT}$.
This gives $(A^{\sT}A)^{\dagger} A^{\sT} = V \Sigma^{-2} V^{\sT}V \Sigma U^{\sT} = A^{\dagger}$. We conclude:

\begin{proposition}
	The set of solution of the minimization problem $\min_{x \in \R^n} \|Ax - y\|^2$ is
	$$
	A^{\dagger} y + \Ker(A).
	$$
\end{proposition}


\section{Penalized least squares: Ridge regression and Lasso}

\section{Norms for matrices}

\section{Low-rank matrix estimation and matrix completion}



\section*{Further reading}


\vspace{1cm}
\centerline{\pgfornament[width=7cm]{71}}


\bibliographystyle{plain}
\bibliography{../references.bib}
\end{document}
