\documentclass{beamer}

\usepackage{../../../latex_style/beamerthemeExecushares}
\usepackage{../../../latex_style/notations}

\title{About the Lagrangian}
\subtitle{Optimization and Computational Linear Algebra for Data Science}
\author{Léo Miolane}
\date{}

\setcounter{showSlideNumbers}{1}

\begin{document}
\setcounter{showProgressBar}{0}
\setcounter{showSlideNumbers}{0}

\frame{\titlepage}
\setcounter{framenumber}{0}
%\setcounter{showProgressBar}{1}
\setcounter{showSlideNumbers}{1}



\begin{frame}[t]{Lagrange multipliers}
	\grid

	\vspace{-0.5cm}
	$$
	\text{minimize}  \qquad f(x) \qquad \text{subject to} \qquad g(x) = 0.
	$$


\end{frame}

\begin{frame}[t]{Encoding the constraint in $L$}
	\grid

	\vspace{-0.2cm}
	We have $L(x,\lambda) = f(x) + \lambda g(x)$, hence
	$$
	\max_{\lambda \in \R} L(x,\lambda) = 
	\begin{cases}
		f(x) & \text{if} \quad g(x) = 0 \\
		+\infty & \text{otherwise}.
	\end{cases}
	$$



\end{frame}

\begin{frame}[t]{Primal and dual problems}
	\grid

	\vspace{-0.2cm}
	\begin{itemize}
		\item Primal optimization problem:
			$$
			\text{minimize} \quad \max_{\lambda \in \R} \ L(x,\lambda) \quad \text{subject to} \quad x \in \R^n.
			$$
		\item Dual optimization problem:
			$$
			\text{maximize} \quad \min_{x \in \R^n} \ L(x,\lambda) \quad \text{subject to} \quad \lambda \in \R.
			$$
	\end{itemize}

\end{frame}

\begin{frame}[t]{Saddle point interpretation}
	\grid

	\vspace{-0.2cm}

	\begin{theorem}
		Assume that:
	\begin{itemize}
		\item $x^{\star}$ is a solution to the primal problem.
		\item $\lambda^{\star}$ is a solution to the dual problem.
		\item Strong duality holds: $\min_x \max_{\lambda} L(x,\lambda) = \max_{\lambda} \min_{x} L(x,\lambda)$.
	\end{itemize}
	\vspace{0.2cm}
	Then $(x^{\star},\lambda^{\star})$ is a saddle-point of the Lagrangian $L$.
	\end{theorem}

\end{frame}


\end{document}
