\documentclass{beamer}

\usepackage{../../../latex_style/beamerthemeExecushares}
\usepackage{../../../latex_style/notations}

\title{Session 10: Linear regression}
\subtitle{Optimization and Computational Linear Algebra for Data Science}
\author{Léo Miolane}
\date{}

\setcounter{showSlideNumbers}{1}

\begin{document}
\setcounter{showProgressBar}{0}
\setcounter{showSlideNumbers}{0}

\frame{\titlepage}
\setcounter{framenumber}{0}
%\setcounter{showProgressBar}{1}
\setcounter{showSlideNumbers}{1}

\begin{frame}
	\frametitle{Contents}
	\begin{enumerate}
		\item Ordinary least squares
		\item Penalized linear regression
		\item Matrix norms
	\end{enumerate}
\end{frame}

\begin{frame}[t]{Introduction}
	\grid

	\vspace{-0.2cm}
	\begin{itemize}
		\item We have $n$  \ << feature vectors >> \ \ \ $a_1, \dots, a_n \in \R^d$.
		\item Each point $a_i$ comes with a \ << target variable >> \ \ \ $y_i \in \R$.
	\end{itemize}

\end{frame}


\begin{frame}[t]{Solving $Ax=y$ is a bad idea}
	\grid

	The system $Ax = y$ may have:
	\begin{itemize}
		\item No solution.
			\vspace{3cm}
		\item Infinitely many solutions.
	\end{itemize}

\end{frame}

\section{Ordinary least squares}

\begin{frame}[t]{Least squares problem}
	\grid

	\vspace{-0.6cm}
	\begin{exampleblock}
		$$
		\textbf{(LS)} \quad \text{Minimize} \quad f(x) = \| Ax - y \|^2 \quad \text{with respect to} \quad x \in \R^d.
		$$
	\end{exampleblock}

\end{frame}
\begin{frame}[t]{The Moore-Penrose pseudo-inverse}
	\grid

	\vspace{-0.4cm}
	\begin{definition}
		Let $A = U \Sigma V^{\sT}$ be the SVD of $A$.
		The matrix $A^{\dagger} \defeq V \Sigma' U^{\sT}$ is called the (Moore-Penrose) pseudo-inverse of $A$, where $\Sigma'$ is the $d \times n$ matrix given by 
		\vspace{-0.4cm}
		$$
		\Sigma'_{i,i} =
		\begin{cases}
			1 / \Sigma_{i,i} & \text{if} \ \Sigma_{i,i} \neq 0, \\
			0 & \text{otherwise},
		\end{cases}
		\vspace{-0.2cm}
		$$
		and $\Sigma'_{i,j} = 0$ for $i \neq j$.
	\end{definition}

\end{frame}

\begin{frame}[t]{Solving $A^{\sT} A x = A^{\sT} y$}
	\grid

	\vspace{-0.1cm}
	\textbf{Claim:} The vector \ $x^{\rm LS} \ \defeq  \ A^{\dagger} y$ \ is a solution of $A^{\sT} A x = A^{\sT} y$

	\vspace{2cm}

	\begin{block}{Theorem}
		The set of the minimizers of \ $f(x) = \|Ax - y\|^2$ \ is
		$$
		A^{\dagger} y + \Ker(A) = \Big\{ x^{\rm LS} + v \, \Big| \, v \in \Ker(A) \Big\}.
		$$
	\end{block}
\end{frame}

\section{Penalized least squares}

\begin{frame}[t]{Ridge regression}
	\grid

	\vspace{-0.2cm}
	Ridge regression consists in adding a << $\ell_2$ penalty >> :
	\vspace{-0.5cm}
	\begin{exampleblock}
		$$
		\textbf{(Ridge)} \quad \text{Minimize} \quad f(x) = \| Ax - y \|^2 + \lambda \|x\|^2 \quad \text{w.r.t.} \quad x \in \R^d
		$$
	\end{exampleblock}
	for some fixed $\lambda > 0$.

\end{frame}

\begin{frame}[t]{Lasso}
	\grid

	\vspace{-0.2cm}
	The Lasso adds a << $\ell_1$ penalty >> :
	\vspace{-0.5cm}
	\begin{exampleblock}
		$$
		\textbf{(Lasso)} \quad \text{Minimize} \quad f(x) = \| Ax - y \|^2 + \lambda \|x\|_1 \quad \text{w.r.t.} \quad x \in \R^d
		$$
	\end{exampleblock}
	for some fixed $\lambda > 0$.

\end{frame}

\begin{frame}[t]{Intuition behind feature selection}
	\grid

	\vspace{-0.4cm}
	\begin{block}{Lemma}
		Let $x^{\rm Lasso}$ be a minimizer of the Lasso cost function and let $r=\|x^{\rm Lasso}\|_1$. Then $x^{\rm Lasso}$ is a solution to the constrained optimization problem:
		$$
		\text{minimize} \quad \|Ax-y\|^2 \quad
		\text{subject to} \quad \|x\|_1 \leq r.
		$$
	\end{block}

\end{frame}

\begin{frame}[t]{Application: compressed sensing}
	\grid

	\only<1>{
		\begin{itemize}
			\item In homework 4 we have seen that we can compress images very well.
			\item Most of the data can be thrown away !
		\end{itemize}
	}
	\pause

\end{frame}


\section{Matrix norms}

\begin{frame}[t]{Frobenius norm}
	\grid

	\begin{block}{Definition}
		The Frobenius norm of a matrix $A \in \R^{n \times m}$ is defined as
		$$
		\|A\|_F = \sqrt{\sum_{i=1}^n \sum_{j=1}^m A_{i,j}^2}
		$$
	\end{block}
	\begin{block}{Proposition}
		$$
		\|A\|_F = \sqrt{\sum_{i=1}^{\min(n,m)} \sigma_i(A)^2}
		$$
	\end{block}
\end{frame}
\begin{frame}[t]{The spectral norm}
	\grid

	\vspace{-0.4cm}
	\begin{block}{Definition}
		The spectral norm of a matrix $A \in \R^{n \times m}$ is defined as
		$$
		\|A\|_{\Sp} = \max_{\|x\| = 1}	\|Ax\|.
		\vspace{-0.3cm}
		$$
	\end{block}
	\begin{block}{Proposition}
		\vspace{-0.3cm}
		$$
		\|A\|_{\Sp} = \sigma_1(A).
		$$
	\end{block}
\end{frame}

\begin{frame}[t]{The nuclear norm}
	\grid

	\vspace{-0.4cm}
	\begin{block}{Definition}
		The nuclear norm of a matrix $A \in \R^{n \times m}$ is defined as
		$$
		\|A\|_{\star} = \sum_{i=1}^{\min(n,m)} \sigma_i(A).
		\vspace{-0.3cm}
		$$
	\end{block}
\end{frame}

\begin{frame}[t]{Application to matrix completion}
	\grid

	\only<1>{
		We have a data matrix $M \in \R^{n \times m}$ that we only observe partially. That is we only have access to 
		$$
		M_{i,j} \quad \text{for} \ (i,j) \in \Omega,
		$$
		where $\Omega \subset \{1, \dots, n\} \times \{1, \dots m\}$ is a subset of the complete set of the entries.
	}

	\pause

\end{frame}

\appendix
\backupbegin
\begin{frame}[t]
	\frametitle{Questions?}
	\grid

	\pause
\end{frame}
\backupend




\end{document}
